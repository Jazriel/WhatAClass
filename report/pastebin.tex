\section{Objetivos técnicos en cuanto al proceso de desarrollo}
\begin{list}{-}{}
\item Para el control de dependencias y despliegue del proyecto y de las mismas se usará pip (freeze e install -r)
\item Como sistema de control de versiones se utilizará git y github
\item El proceso de CI se facilitará con travis y codeclimate
\item Para ayudar con el feedback (retroalimentación) y el desarrollo continuo se usará Zenhub sobre github y slack como herramienta para mantener contacto con los 'clientes' (en este caso los tutores)
\end{list}

__

 Este problema emerge ya que la complejidad de los sistemas aumenta excesivamente si no se hacen esfuerzos por desacoplar subsistemas
__


%____________
Refactorizaciones importantes : patron factory, configuracion como un archivo y variables de entorno y blueprints más separadas.

%___________


Tras pagina web -> pipeline de desarrollo
la pipeline de desarrollo se buscó un sistema flexible lo cual fue conseguido con git en github enlazandolo con travis para una integración continua y con otras herramientas como slack para comunicación con el resto del equipo.

CodeClimate fue usado como la herramienta elegida para comprobar la calidad del código y el recubrimiento de los test. VersionEye para el control de versiones.

%____________
Refactorizaciones importantes : patron factory, configuracion como un archivo y variables de entorno y blueprints más separadas.

%___________


Tras pagina web -> pipeline de desarrollo
la pipeline de desarrollo se buscó un sistema flexible lo cual fue conseguido con git en github enlazandolo con travis para una integración continua y con otras herramientas como slack para comunicación con el resto del equipo.

CodeClimate fue usado como la herramienta elegida para comprobar la calidad del código y el recubrimiento de los test. VersionEye para el control de versiones.
