\capitulo{6}{Conclusiones y líneas de trabajo futuras}

En esta sección se exponen los resultados de la experiencia con el trabajo y lineas futuras con las que mejorar y dar continuidad al proyecto.

\section{Conclusiones}

Tras trabajar en un proyecto de tanta envergadura puedo decir sin temor a equivocarme que ha sido demasiado amplio, por culpa de mi propia ambición, tocando muchos temas de los cuales no salimos suficientemente preparados de la carrera:

\begin{itemize}
\item \textbf{Desarrollo web:} Es una de las partes de las que más trabajo tienen ahora mismo, debido a la cantidad de empresas que quieren tener una página web. Acabamos con una mínima preparación en C\#, y sin suficiente conocimiento de la infraestructura física (servidores, \eng{switches}, \eng{routers}...), ni lógica (balanceadores de carga, \eng{reverse proxies}, servidores web...) que se necesitan para poner una página web en funcionamiento.
\item \textbf{Redes neuronales convolucionales profundas:} Tras cursar la asignatura de Computación neuronal y evolutiva destacó el desconocimiento de como modelar redes neuronales y que las convolucionales, que ahora mismo parecen el futuro de campos como la conducción automática, reconocimiento y generación de imágenes..., casi ni se mencionasen.
\end{itemize}

Otros temas como la virtualización de los entornos de despliegue no parece tan importante, ya que la mayoría de las aplicaciones pueden desarrollarse y desplegarse cómodamente sin este tipo de servicios. Quizá valga la pena cierta mención de este tipo de soluciones con las ventajas e inconvenientes de contenedores y maquinas virtuales. 

Personalmente pienso que la orquestración de servicios sin tener el hardware necesario es demasiado difícil. A pesar de su dificultad probablemente acabe siendo necesario el conocimiento de este tipo de sistemas para el desarrollo de software del tipo \eng{Software as a Service} (SaaS), otra alternativa a este tipo de servicios es usar la `nube', de empresas como heroku, amazon (aws)...

El proyecto ha llegado hasta un punto en el que puedo decir, que estoy satisfecho con lo conseguido, y a pesar de tener que recortar en ciertos aspectos como la investigación relacionada con el \eng{deep learning}, el conocimiento adquirido durante este periodo parece invaluable. Por último querría recomendar a cualquiera que lea este trabajo que tenga cuidado con uno de los problemas más grandes con el desarrollo ágil, el \eng{scope creep}, esta es la idea de que, según trabajamos en un proyecto, surjan nuevas e interesantes lineas de desarrollo, investigación o oportunidades y al perseguirlas acabemos ampliando el enfoque del proyecto a cubrir más temas, sin tener en cuenta si esto es suficientemente beneficioso o implica el recortar de otras partes del proyecto.
 

\section{Líneas de trabajo futuras}
Las lineas de trabajo futuras son más o menos claras:

\begin{itemize}
\item Nginx: Estudiar Nginx y su configuración como balanceador de carga, \eng{reverse proxy} y servidor de http y https. 
\item Redis: Investigar sobre servicios similares a Redis y memcache que, básicamente, mejoran la eficiencia de uso de una base de datos mediante el cache en memoria principal de las consultas realizadas.
\item NoSQL: No se recomienda usar NoSQL como fuente de escalabilidad a no ser que sea el único lugar del cuál podemos aumentar el rendimiento, esto se debe a que páginas como Facebook usan MySQL para los usuarios, las bases de datos NoSQL son para cargas mucho más pesadas, como estadísticas captadas en tiempo real. Algunas posibilidades son Apache Cassandra (\hfoot{https://pycassa.github.io/pycassa/}{Pycassa}) y MongoDB (\hfoot{https://api.mongodb.com/python/current/}{Pymongo}).
\item DNS: Se cree necesario investigar cómo hacer balance de carga a nivel de DNS, porque es una forma aparentemente sencilla de escalar una aplicación geográficamente, es decir permitir balance de carga según el tiempo de respuesta de distintos servidores que tengamos a nuestra disposición.
\end{itemize}

