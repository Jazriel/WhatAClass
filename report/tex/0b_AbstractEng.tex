Recently, most software centered companies are trying to improve their services with an 'AI first' approach, this is more prevalent in large companies trying to apply Marchine Learning more widely. 

Scalability is a concept with a clear but very abstract definition \cite{hill90}, the best definition that has been found is attributed to Bondi \cite{bondi00} who defined it as the ability of a system to be able to expand to handle an increasing work load. 

From the point of view of maintenance, a scalable system is that can be maintaned or whose functionality can be increased without incurring in a technical debt too large, allowing a faster development. 

Scalability is a complex property in the software world as there are several places where a system can be expanded. The most common way of providing scalability is by increasing the number of copies of running software, on the same computer or on different computers. Another approach it to divide the system into the smallest subsystems possible so that the complexity decreases making the system easier to maintain. Finally, it is possible to separate the services in different partitions so that each partition is in charge of providing part of the results.

These three scaling options are called the `\emph{cube of scalability}' in the book \emph{The Art of Scaling}\cite{scala09}. 

In this work we seek to scale a system using the first two scaling axes. We will achieve the scaling in the first axis or dimension by using containers, the system we will use is docker. The scaling in the second dimension will be obtained with the microservice architecture.

This seeks to facilitate access to the solution of a common problem in the field of Machine Learning that is the image classification. This is done by using Tensorflow. This way of proceed gives an example of one of the ways to avoid scalability problems, a quite common problem in the world of software development. This scalability is both in performance and maintenance. It seeks to provide an architecture easily scalable in both directions that avoid increasing the complexity of the code to facilitate its future reuse and learning. It also attempts to provide a skeleton so that it can be used by future projects that need to use a basic user control, investing the time released in other parts or topics of greater interest. 

To illustrate how the challenge of adding more services is solved, a service that uses Deep Learning has been implemented to classify images on a set of classes, such as Imagenet \cite{imnet}. 
