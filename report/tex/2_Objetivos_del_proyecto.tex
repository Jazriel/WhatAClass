\capitulo{2}{Objetivos del proyecto}

A continuación se indican los objetivos, tanto teóricos marcados por los requisitos como objetivos que se persiguen con el proyecto. También se incluyen ciertos requisitos que no tienen por que ser obvios pero que en la actualidad se esperan de cualquier aplicación web.



\section{Objetivos principales} 
\begin{list}{-}{}
\item Conseguir un sistema actual y escalable.
\item Seguir principios de desarrollo actualizados.
\item Proporcionar control de usuarios a aquella persona que lo necesite y su proyecto encaje con el que aquí se muestra.
\end{list}

\section{Servidor web y página web}
\begin{list}{-}{}
\item Arquitectura MVC (Model View Controller)
\item Facilidad de uso: Que el diseño sea intuitivo y fácil de aprender a usar.
\item Internacionalización: Preparar la aplicación para que este disponible en varios idiomas.
\item Sistema responsivo: Que se adecue al dispositivo desde el que se visita.
\end{list}

\section{Servicio de Minería}


\section{Proceso de desarrollo}
Los puntos siguientes se ven como necesidades fundamentales en cuanto al desarrollo de proyectos actualmente a los que quizá no se de suficiente importancia.
\begin{list}{-}{}
\item Sistema para el control de dependencias
\item Despliegue de el proyecto
\item Sistema de control de versiones
\item Tener un proceso de CI (Continuous integration) 
\item Programar con agilidad
\end{list}

\section{Objetivos personales}
\begin{list}{-}{}
\item Aprender arquitecturas actuales, las cuales se pueden usar tanto en industria como en academia.
\item Avanzar mis conocimientos a partir de los obtenidos en la carrera, sobre todo aquellos que tienen importancia real y quizá no se hayan estudiado suficiente en la carrera
\item Profundizar en el entorno de Python, ya que en los últimos años se ha incrementado su importancia tanto para \textit{Data Scientists} como para desarrolladores web, que son las dos profesiones encuentro muy interesantes.
\end{list}

