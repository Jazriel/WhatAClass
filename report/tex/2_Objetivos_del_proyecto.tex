\capitulo{2}{Objetivos del proyecto}

A continuación se indican los objetivos, tanto teóricos marcados por los requisitos como objetivos que se persiguen con el proyecto. También se incluyen ciertos requisitos que no tienen por que ser obvios pero que en la actualidad se esperan de cualquier aplicación web.



\section{Objetivos principales} 
\begin{list}{-}{}
\item Facilitar investigacion y explotación de sistemas de machine learning.
\item Unificar una interfaz para interactuar con sistemas de machine learning.
\item Posibilitar el uso de las características anteriores mediante una aplicación web.
\end{list}

\section{Objetivos parte mineria}
\begin{list}{-}{}

\item Facilidad de incremento en el numero de sistemas de machine learning 
\item Flexibilidad en los posibles sistemas de machine learning
\item Flexibilidad en el origen de los sistemas de machine learning
\item Extraccion de datos desechables para investigacion
\item Flexibilidad en los datasets introcucibles

\end{list}

\section{Objetivos parte web app}
\begin{list}{-}{}
\item Arquitectura MVC (Model View Controller)
\item Facilidad usable
\item Internacionalizacion posible
\item Responsiveness
\end{list}

\section{Objetivos proceso de desarrollo}
\begin{list}{-}{}
\item Usar algún sistema para el control de dependencias y despliegue de el proyecto y dependencias
\item Utilizar un sistema de control de versiones
\item Tener un proceso de CI (Continuous integration) 
\item Programar con agilidad, no agilmente 
\item Programar siguiendo el Manifesto for Software Craftsmanship
\item Distribuir el proyecto con alguna herramienta de fácil uso
\end{list}

\section{Objetivos técnicos en cuanto al proceso de desarrollo}
\begin{list}{-}{}
\item Para el control de dependencias y depliegue del proyecto y de las mismas se usará pip (freeze e install -r)
\item Como sistema de control de versiones se utilizará git y github
\item El proceso de CI se facilitará con travis y codeclimate
\item Para ayudar con el feedback (retroalimentación) y el desarrollo continuo se usará Zenhub sobre github y slack como herramienta para mantener contacto con los 'clientes' (en este caso los tutores)
\end{list}

\section{Objetivos personales}
\begin{list}{-}{}
\item Crear una herramienta que facilite lineas de investigacion poco exploradas
\item Avanzar mis conocimientos a partir de los obtenidos en la carrera, sobre todo en los campos cuya relacción importancia real, conocimiento obtenido es mayor
\item Profundizar en el entorno de Python, ya que en los ultimos años se ha puesto de moda.
\end{list}

