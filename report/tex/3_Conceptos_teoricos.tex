\capitulo{3}{Conceptos teóricos}

Algunos conceptos teóricos tanto de la parte técnica como de la parte de procesos de software


\section{DevOps}

DevOps es un acrónimo inglés de: 'software \textbf{Dev}elopment and information technology \textbf{Op}eration\textbf{s}' es un termino que engloba un conjunto de prácticas de colaboración y comunicación entre desarrolladores software y técnicos informáticos. Los objetivos de esta comunicación y colaboración son una construcción de software más consistente y confiable. Este proceso heredero de las técnicas ágiles se basa en una \textit{cadena de herramientas}. Esta cadena de herramientas es algo que no esta completamente definida pero más o menos la podemos concretar, cabe tener en cuenta que esta cadena cambia según a quien le preguntes.

\subsection{La 'cadena de herramientas' de DevOps}

Esta cadena de herramientas se basa en siete procesos con sus correspondientes herramientas:

\begin{enumerate}
 \item Plan: Consistente en determinación de metricas, requerimientos\ldots y una vez pasemos de la primera iteración ha de tener en cuenta el feedback del cliente.
 \item Creación: Es el proceso de programar y crear el software, las herramientas en este proceso es el software de control de versiones que vayamos a usar.
 \item Verificación: Proceso de comprobación de la calidad del software. Normalmente consiste en hacer test de diversos tipos (Aceptación, seguridad\ldots)
 \item Preproducción o empaquetación: En esta fase se piden aprobaciones de los distintos equipos y se configura el paquete.
 \item Lanzamiento: En este punto se prepara el horario de lanzamiento y se orquesta el software para poder ponerlo en el entorno de producción objetivo.
 \item Configuración: Una vez el software esta desplegado toda la parte de la infraestructura y configuración de la misma se incluye en esta categoría, como por ejemplo las bases de datos, configuración de las mismas\ldots 
 \item Monitoreo: Tras entregar el software se mide su rendimiento en la infraestructura objetivo y se mide la satisfacción del usuario final. Se recogen metricas y estadísticas
\end{enumerate}



