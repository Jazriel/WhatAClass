\capitulo{5}{Aspectos relevantes del desarrollo del proyecto}

%Este apartado pretende recoger los aspectos más interesantes del desarrollo del proyecto, comentados por los autores del mismo.



%Debe incluir desde la exposición del ciclo de vida utilizado, hasta los detalles de mayor relevancia de las fases de análisis, diseño e implementación.

Ciclo de vida


Primero la pagina web
Análisis y diseño -> MVC, referencias 
Implementacion tutorial en python2 a python3 solo ligeras diferencias pero algo difíciles de encontrar.
Se decidió refactorizar al acabar este punto. Introduciendo un patrón factoría que a pesar de ser recomendado en la documentación oficial supuso gran cantidad de dificultades, lo cual me hace pensar que haya sido una decisión errónea. Ya que las ventajas de poder tener el paquete importado sin el esfuerzo de crear la aplicación.

Tras pagina web -> pipeline de desarrollo
la pipeline de desarrollo se buscó un sistema flexible lo cual fue conseguido con git en github enlazandolo con travis para una integración continua y con otras herramientas como slack para comunicación con el resto del equipo.

CodeClimate fue usado como la herramienta elegida para comprobar la calidad del código y el recubrimiento de los test. VersionEye para el control de versiones.


Llegados a este punto pareció interesante desplegar el proyecto en un sitio web. Se eligió heroku. Para facilitar el uso de heroku se hizo un cambio a como funcionaba la configuración (refactorización). 

Docker se empezó a usar docker a continuación, la adaptación fué compleja y costosa debido a un desconocimiento total de la tecnología. Al principio se consiguió meter la aplicación en un contenedor de manera simple.


%Se busca que no sea una mera operación de copiar y pegar diagramas y extractos del código fuente, sino que realmente se justifiquen los caminos de solución que se han tomado, especialmente aquellos que no sean triviales.



%Puede ser el lugar más adecuado para documentar los aspectos más interesantes del diseño y de la implementación, con un mayor hincapié en aspectos tales como el tipo de arquitectura elegido, los índices de las tablas de la base de datos, normalización y desnormalización, distribución en ficheros3, reglas de negocio dentro de las bases de datos (EDVHV GH GDWRV DFWLYDV), aspectos de desarrollo relacionados con el WWW...



%Este apartado, debe convertirse en el resumen de la experiencia práctica del proyecto, y por sí mismo justifica que la memoria se convierta en un documento útil, fuente de referencia para los autores, los tutores y futuros alumnos.
