\capitulo{1}{Introducción}


El uso mayoritario del aprendizaje automático o machine learning (ML) se puede ver en empresas como Google que ha pasado de buscar el termino exacto pedido por el usuario a intentar, con la información de que disponen de averiguar en que contexto se esta buscando por ejemplo si predice que eres programador y buscas R probablemente los primeros resultados sean del lenguaje de programación.

El servicio a dar acceso es un sistema de machine learning y no de otra clase, por la actualidad y relevancia de la tecnología. El sistema de aprendizaje automático será un clasificador que agrupará las imagenes en los conjuntos de Imagenet [ref http://image-net.org/].

La escalabilidad es muy importante en el desarrollo software actualmente debido a la necesidad de hacer sistemas cada vez más grandes y complejos que a la vez no gasten recursos de manera excesiva. Esto se puede conseguir con ciertas arquitecturas como microservicios. Otras arquitecturas que proporcionan este tipo de ventajas son Serverless[ref https://serverless.com/  y serverless ops].

Este trabajo se orienta a conseguir un sistema mediante métodos actuales y escalable. Se intentará que los métodos para conseguir este sistema sean lo más actuales, prestigiosos y usados posibles, ya que probablemente esos métodos serán fundamentales el día de mañana. También se persigue adquirir los conocimientos necesarios para poder replicar este sistema sobre proyectos ya creados.

El sistema en cuestión será una página web, ya que es un sistema altamente accesible (desde casi cualquier plataforma) con mayor facilidad de mantenimiento que la mayor parte de aplicaciones especificas a un dispositivo concreto. Otra ventaja que proporcionan las páginas web es que tendremos acceso a todos los errores y fallos que surjan en ejecución.

Se usarán tecnologías puntas para conseguir estos objetivos. Como docker para la escalabilidad y reducción de deuda técnica gracias a la arquitectura de microservicios que facilita.

La metodología que se usará será la integración continua ya que permite reducir la deuda técnica derivada de la integración de distintos servicios entre sí. Esta consiste básicamente en intentar integrar los servicios que se disponen a cada paso que se da en la creación de software. En proyectos grandes esto podría ser cada día y en proyectos pequeños cada commit.
