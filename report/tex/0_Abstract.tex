Recientemente la mayoría de empresas que usan productos informáticos están intentando mejorar sus servicios con un enfoque `AI first', esto es más prevalente en los grandes del sector que  tratan de aplicar el aprendizaje automático de manera más amplia.

La escalabilidad es un concepto con una definición clara pero muy abstracta \cite{hill90}, la mejor definición que se ha encontrado se atribuye a Bondi \cite{bondi00} y la define como la capacidad de un sistema para ser capaz de ampliarse para manejar mayor cantidad de trabajo. 

Desde el punto de vista de mantenimiento un sistema escalable es aquel al que podemos mantener o incrementar su funcionalidad sin incurrir en una cantidad de deuda técnica demasiado grande permitiendo un desarrollo más rápido.

La escalabilidad es una propiedad compleja en el mundo software ya que existen varias fuentes de ampliación del sistema, la más común es la escalabilidad aumentando el número de copias del software que se ejecutan, en el mismo ordenador o en distintos ordenadores. Se puede descomponer el sistema en los subsistemas más pequeños posibles de manera que la complejidad disminuye convirtiendo el sistema más fácil de mantener. También existe una última que es separar los servicios en particiones distintas de manera que cada partición se encarga de dar servicio a parte de los datos.

Estas tres opciones de escalado se conocen en el libro The Art of Scaling \cite{scala09} como el `cubo de la escalabilidad'. 

En este trabajo se busca obtener un sistema escalable en los dos primeros ejes de escalado. El primer eje o dimensión lo vamos a conseguir con contenedores, el sistema que usaremos es docker. La segunda dimensión la obtendremos con la arquitectura de microservicios. 

Esto busca dar un fácil acceso a un problema común en el campo del aprendizaje automático que es la clasificación de imágenes. Esto se hace mediante Tensorflow.

Con esto se pretende dar un ejemplo de una de las  maneras de evitar problemas de escalabilidad, un problema bastante común en el mundo del desarrollo software. Esta escalabilidad es tanto de rendimiento como de mantenimiento. Se busca facilitar una arquitectura fácilmente escalable en ambos sentidos cuyo reflejo en código no es complejo para el futuro aprendizaje. También se intenta proporcionar un esqueleto para que los futuros proyectos que quieran usar un control básico de usuarios puedan usar para centrarse en otras partes o temas de mayor interés.

Para exponer cómo se resuelve el añadir más servicios se ha expuesto un servicio que usa Deep Learning para clasificar imágenes sobre un conjunto de clases (Imagenet \cite{imnet}).