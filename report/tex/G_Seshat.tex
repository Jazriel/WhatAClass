\apendice{Subproducto: Seshat}

Como subproducto del proyecto se ha integrado la autenticación de usuarios a el proyecto Seshat en el repositorio \hfoot{https://github.com/cgosorio/SeshatAuth}{SeshatAuth}. Este repositorio es privado y se necesita permiso de acceso, aun así se incluye un clon del mismo en las copias digitales por facilidad de acceso.

Este subproducto también se ha subido de forma minimizada al repositorio \hfoot{https://github.com/Jazriel/auth-that}{auth-that}. 

\section{Estructura de directorios}
La estructura de directorios se ha mantenido, simplemente se han añadido los archivos necesarios. En la carpeta Servidor están todos los archivos añadidos, se comentan solo aquellos archivos que han sido modificados o añadidos.


\begin{tabbing}
\hphantom{tab }\= \hphantom{tab }\= \hphantom{tab }\= \hphantom{tab }\= \hphantom{quadruple tabula}\= \kill\\
config/default.py \> \> \> \> \> \textit{Archivo donde se almacena la configuración} \\
src/ \> \> \> \> \> \\
\> server/ \> \> \> \> \\
\> \> templates/ \> \> \> \textit{Se ha modificado el formulario maestro para poder devolver mensajes} \\
\> \> \> \> \> \textit{Se ha modificado el formulario maestro para poder devolver mensajes} \\
\> user\_control/ \> \> \> \> \textit{Localización del código} \\

\> user\_server.py \> \> \> \> \textit{Principal archivo donde se integra la autenticación.} \\
test/ \> \> \> \> \> \\
.gitignore \> \> \> \> \> \textit{Se ignoran los archivos .idea} \\
debug.py \> \> \> \> \> \textit{Ejecución en depuración} \\
requirements.txt \> \> \> \> \> \textit{Cambio de las dependencias} \\
run.py \> \> \> \> \> \textit{Ejecución en producción} \\
\end{tabbing}

\section{Manual del programador}

La serie de pasos para pasar de tener el subproducto auth-that y Seshat se detallan a continuación, como consideración los dos proyectos en un principio deberían estar independientes:

\begin{enumerate}
\setlength{\itemsep}{1pt}
\setlength{\parskip}{0pt}
\setlength{\parsep}{0pt}
\item En auth-that: refactorizar project\_name/ a  src/, en otros proyectos serían otros nombres.
\item Eliminar los archivos `placeholder' como `src/\_\_init\_\_.py' y `src/project.py'
\item Copiar todo el proyecto auth-that a la raíz del código fuente otro proyecto.
\item Cambiar la importación en `user\_server.py' para que importe la app de flask de la otra aplicación.
\item Configurar la base de datos(uri), servidor de email y \textit{token} de Google. Esto se puede hacer en `config/default.py' o `src/user\_server.py'.
\end{enumerate}

Generalmente no debería haber problemas, pero debido a que no se pueden asumir casi nada de que tenía la aplicación que hemos importado puede ser que den problemas, principalmente, dos cosas:

\begin{itemize}
\setlength{\itemsep}{1pt}
\setlength{\parskip}{0pt}
\setlength{\parsep}{0pt}
\item Extensiones de Flask iguales
\item Rutas iguales
\end{itemize}

Aunque normalmente las extensiones de Flask no tendrían porque dar problemas podrían hacerlo. Tristemente no hay una solución fácil y uniforme para esta clase de problema. Hay que arreglarlo a mano.

Las rutas hay que comprobarlas manualmente, o podemos añadir un prefijo al subproducto de autenticación en `user\_server.py':


\lstset{style=blockstyle}
\begin{lstlisting}[language=Python]
from .user_control.controllers import user_mng

app.register_blueprint(user_mng)
# Se cambia esta linea por:
app.register_blueprint(user_mng, url_prefix='/user')
# O
app.register_blueprint(user_mng, subdomain='user')
\end{lstlisting}

La primera direccionará las rutas del tipo: `\textit{pagina.com/user/login}'

La segunda lo hará de la siguiente manera: `\textit{user.pagina.com/login}'





